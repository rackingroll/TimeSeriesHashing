\section{The Approach}
\label{sec:framework}
In this section, we first propose a framework to analyze the correlation of heterogeneous time series, and then we introduce how to use hashing method to do fast searching and clustering tasks. 

\subsection{Change Based Correlation Coefficient}

As we introduced in Section. \ref{sec:formulation}, change information is important for calculating the change based correlation method. 
So, Given a time series, the first thing need to do is to extract the change information of the time series. In this work, the change information is a bit-stream.
After obtaining the change information of the time series, we then calculate the Jaccard similarity coefficient between each other. The Jaccard similarity between each bit-stream will be the correlation coefficient between these two time series. 
The framework of this correlation is showed in Fig.\ref{FarameworkOverview}.
 
\begin{figure}[t]
\centering
\includegraphics[width=0.3\textwidth]{temp.pdf}
\caption{Overview of the Framework}
\label{FarameworkOverview}
\end{figure}

\subsection{Change Information Extraction}
\label{ChangeCorrelation}

As we introduced in Section.\ref{sec:formulation}, change based correlation corresponds to the change information of the time series. Change information of a time series is a time period information, not a time point information. As a result, in order to extract the change information of the time series, we need to find the information in small time period of the time series (A sub-series). 

Given a time series $S = (s_1,s_2,...,s_m)$, where $m$ is the number of points in the time series. Given a subs-series length $k$. The change information of the time series $S$ can be represented as a bit-stream:

$B_S = \{b_0,b_1,...,b_n\}$, where each $b_i$ corresponds to a sub-series of length $k$ for the original time series $S$ as showed in Fig.\ref{ChangeMapping}.

Given a sub-series $l^j = \{s_i,s_{i+1},s_{i+2},...,s_{i+k-1}\}$, where $w$ is the length of the sub-series. Then the change information of sub-series $l$ is denoted as follow:

\begin{equation}
\label{Equ:ChangeInformation}
b^l = \left\{\begin{matrix}
1 & Have~change~in~l
\\ 
0 & No~change~in~l
\end{matrix}\right.
\end{equation}

\begin{figure}[t]
\centering
\includegraphics[width=0.3\textwidth]{temp.pdf}
\caption{Change Information Extraction}
\label{ChangeMapping}
\end{figure}

As showed in Equ.\ref{Equ:ChangeInformation}, the change information is the information that whether there's a change in the sub-series. In order to denote whether there's a change in the sub-series, we need to know to to detect change in the sub-series.

\subsection{Change Detection}

So,the problem here is:
Given a sub-series 

$l^j = \{s_i,s_{i+1},s_{i+2},...,s_{i+k-1}\}$, 

how to denote whether there is a change or not in this time series.
There are a so many time series change detection methods \cite{liu2013change,chen2013contextual} proposed in the literature. 

In this work, the change detection task here is not find the change points of the time series, instead we only need to denote whether there's a change in the time series. It is pointed out that all the change point detection methods can be used here to detect change information.
In our experiment, we use the the following method to detect change:

We equally divide the time series into two series: 

$l^j_{Front} = \{s_i,s_{i+1},s_{i+2},...,s_{i+(k-1)/2 -1}\}$

and, 

$l^j_{Front} = \{s_{i+(k-1)/2 -1},s_{i+(k-1)/2},...,s_{i+k-1}\}$.

So, regard $l^j_{Front}$ and $l^j_{Rear}$ as two data sampled from two distributions $P_1$ and $P_2$. So, if $P_1$ and $P_2$ are statistically the same, then we can say there's not change between each other. Otherwise, there is a change in this dataset.

Then, the problem here becomes a \textit{Two Sample Problem} \cite{gretton2006kernel}. We use the Two Sample \textit{t}-test \cite{moore2007basic} method to solve this problem:

Here, the $t_{score}$ between $l^j_{Front}$ and $l^j_{Rear}$ can be calculated as:

\begin{equation}
t_{score} = \frac{\overline{l^j_{Front}} - \overline{l^j_{Rear}}}{\sigma_p\sqrt{2/k}}
\end{equation}

where, $\overline{l^j_{Front}}$ and $\overline{l^j_{Front}}$ are the mean values of $l^j_{Front}$ and $l^j_{Front}$. And $\sigma_p$ is as follow:

\begin{equation}
\sigma_p = \frac{(k-1)\sigma_{l^j_{Front}}^2 + (k-1)\sigma_{l^j_{Rear}}^2}{k-1}
\end{equation}

Then, if $t_{score} > \alpha$, we can say that these two samples are from different distributions, and thus there is a change in the sub-series $l^j$.

\subsection{Jaccard Similarity Coefficient}

After obtaining the change information (Bit-stream) of each data, we then use Jacord Similarity Coefficient to calculate the Change Correlation of each time series.

The Jaccard Similarity is defined as follow:
Given two Bit-stream $X$ and $Y$, the Jaccord distance is showed as follow:

\begin{equation}
J(A,B) = \frac{|A \cap B|}{|A \cup B|}
\end{equation}

where, $|A \cap B|$ denotes the number of bit that $X$ and $Y$ both $1$. And $|A \cup B|$ denotes the number of bit that at least $X$ or $Y$ is $1$.
For example, given two bit stream: $X={100111}$, and $Y={001110}$. Then $|A \cap B| = 2$, and $|A \cap B| = 5$, so $J(A,B) = \frac{2}{5} = 0.4$.


%\subsection{Speed up Top-k Searching using Hashing}
%
%From section \ref{ChangeCorrelation}, we can see the time series correlation is transfered to the Jaccard Similarity Coefficient. So, we can use LSH method to do fast nearest neighbors search 
%
%
%The overall algorithm is then summarized as Algorithm 1. This algorithm implements the methods we have introduced in the above subsections. It performances two two-sample tests based on the sub-series $\varGamma^{front}$, $\varGamma^{rear}$, and $\varTheta$. The output of this algorithm contains all three aspects of the correlation between a time series and an event sequence. In this section, we discuss some details of the algorithm implementation.
%\begin{algorithm}[t]
%\caption{The Fast Search Algorithm}
%\KwIn{ A set of time series $C=\{S_1, S_2,..., S_n\}$, a Query Time Series $S=(s_1, s_2,..., s_m)$, the sub-series length $w$, and the search parameter $k$.}
%\KwOut{Top k nearest neighbors of $S$ in $C$ }
%
%Normalize each $S_i \in C$ ;
%
%Extract the change information $B_i$ for each time series $S_i$.
%
%Extract the change information $B$ for the Query Time Series $S$.
%%$R = D_f \& D_r$
%
%Do MinHash for each for each bit-stream of time series k.
%
%Out put $R$, $D$ and $T$\;
%
%Algorithm End.
%
%\end{algorithm}
%
%The overall algorithm is then summarized as Algorithm 1. This algorithm implements the methods we have introduced in the above subsections. It performances two two-sample tests based on the sub-series $\varGamma^{front}$, $\varGamma^{rear}$, and $\varTheta$. The output of this algorithm contains all three aspects of the correlation between a time series and an event sequence. In this section, we discuss some details of the algorithm implementation.
%\begin{algorithm}[t]
%\caption{The Fast Clustering Algorithm}
%\KwIn{ A set of time series $C=\{S_1, S_2,..., S_n\}$, Cluster number $t$, the sub-series length $w$, and the search parameter $k$.}
%\KwOut{Top Cluster result }
%
%Normalize each $S_i \in C$ ;
%
%Extract the change information $B_i$ for each time series $S_i$.
%
%Extract the change information $B$ for the Query Time Series $S$.
%%$R = D_f \& D_r$
%
%Do MinHash-Clustering for each for each bit-stream of time series k.
%
%Output Clustering result.
%
%\end{algorithm}

\subsection{Sub-series Length w}

In this research, the sub-series length $w$ is a very important parameter.
If the sub-series length $w$ is too short, then the change information can not be captured. On the other hand, if the sub-series length $w$ is too long, then there will be to much noise information.

In some cases, the value of $k$ can be selected based on domain knowledge and experiments. However, in most real world situations, there are millions of time series and events, and we do not have enough domain knowledge to pre-select the values of all sub-series lengths. In this research, we use the parameter as our previous research work on Correlating Event with time series \cite{luo2014correlating}.
This method can auto-select the sub-series length for a time series based on the autocorrelation function \cite{hamilton1994time} of the time series.
Given a time series $S=(s_1,s_2,...,s_n)$, the autocorrelation is showed as follow:

\begin{equation}
R(l) = E(s_i*s_{i-l}).
\end{equation}
where $l$ denotes the lag of the correlation. The autocorrelation function of a time series can be used to represent the energy of signals in the time series with a period of $l$ \cite{hamilton1994time}. Therefore, our length $k$ can be assigned as the value of the first peak to include the significant signal of the time series. 
